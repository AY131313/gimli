\section{Inversion properties}
An inversion is a template class of the underlying data vector Vec and the matrix type Mat.
It is initialised with one of the following constructors:
\begin{lstlisting}
//! very simple and empty inversion
Inversion( bool verbose = false, bool dosave = false ) 
//! Usual inversion constructor with data and FOP
Inversion( Vec data, ModellingBase & forward, bool verbose, dosave)
//! Complete constructor including transformations
Inversion( Vec data, ModellingBase & forward, 
    Trans & transData, Trans & transModel, bool verbose, bool dosave )
\end{lstlisting}
The properties are not visible itself, instead there are setter and getter functions. Setters:
\begin{lstlisting}
setRelativeError( double/Vec error);  //! set relative data error
setAbsoluteError( double/Vec error);  //! set absolute data error
setForwardOperator( ModellingBase & forward ); //! set forward operator
setTrans( transData, transModel );   //! set transformations
setTransData( transData );           //! set data transformation
setTransModel( transModel );         //! set model transformation
setLineSearch( bool isLinesearch );  //! switch line search on/off
setRobustData( bool isRobust );      //! IRLS (robust) data weighting
setBlockyModel( bool isBlocky );     //! IRLS (blocky) model constraints
setLambda( double lambda );          //! regularisation strength
setOptimizeLambda( bool optLambda ); //! L-curve optimization
setMaxIter( int maxIter );           //! define maximum iteration number
setModel( Vec model );               //! set model vector
setModelRef( Vec referenceModel );   //! set reference model vector
setCWeight( Vec cWeight );           //! set constraint control vector
setMWeight( mWeight );               //! set model control vector
\end{lstlisting}
Getter%Vec error();
\begin{lstlisting}
ModellingBase & forwardOperator();     //! forward operator
uint boundaryCount(), modelCount(), dataCount();//! # boundaries/cells/data
bool lineSearch(), blockyModel(), robustData(), optimizeLambda();//!options
double getLambda();                   //! regularisation strength
int maxIter();                        //! maximum iteration number
Vec model(), response(), roughness(); //! model/response/roughness vector
Vec cWeight(), mWeight();             //! constraint/model control vector
getPhiD(), getPhiM(), getPhi(), getChi2(); //! objective function parts
\end{lstlisting}
Run inversion and other actions
\begin{lstlisting}
Vec model = run();               //! runs the whole inversion
Vec model = oneStep();           //! runs one inversion step
Vec model = runChi1();           //! runs changing lambda such that chi^2=1
robustWeighting();               //! applies robust data weighting
constrainBlocky();               //! apply blocky model contraints
echoStatus();                    //! echo chi2/phi/phiD/phiM/iteration
Vec modelCellResolution( iRes ); //! compute a column of resolution matrix
\end{lstlisting}

\clearpage
\section{Mesh types and mesh generation}\label{app:meshes}
There are different mesh types and ways how to generate them. There is only one base mesh class holding the nodes/vertices/coordinates, cells (defined by the bounding vertices) and boundaries revealing the neighboured cells.
Every node, cell and boundary has a marker that defines the behaviour in modelling (e.g. an electrode node or boundary conditions) or inversion (e.g. the region number or a known sharp boundary).

\subsection*{0d mesh}
Zero dimension means that there are several parameters without any neighbouring relation.
Consequently 0th order constraints are used.
There is no special mesh generator, instead a 1d mesh is created and the constraint type is set to zero.
Alternatively, the forward operator is initialised without mesh and the parameter number is set by setParameterCount.
\subsection*{1d mesh}
A real 1d mesh subdivides the subsurface in vertically or horizontally aligned elements and can be created by the following functions:
\begin{lstlisting}
/*! Generate 1d mesh with nCells cells (size 1) and nCells+1 nodes */
Mesh createMesh1D( uint nCells, uint nClones = 1 );
/*! Generate simple one dimensional mesh with nodes in RVector pos */
Mesh createMesh1D( const RVector & x );
\end{lstlisting}
nClones can be used to create models for different parameters such as resistivity and phase.
Result is one mesh with two sub-meshes that are individual regions.
See section~\ref{sec:dc1dsmooth} for an example.

\subsection*{1d block model}
A 1d block model consists of (nLayers-1) thicknesses and nLayers values for a number of properties.
The thicknesses and properties are individual 1d meshes.
\begin{lstlisting}
/*! Generate 1D block model of thicknesses and properties */
Mesh createMesh1DBlock( uint nLayers, uint nProperties = 1 );
\end{lstlisting}
See section~\ref{sec:dc1dblock} for a resistivity block inversion or \ref{sec:blockjoint} for joint block inversion using two properties.

\subsection*{2d regular mesh}
A regular (FD like) 2d model consists of regularly spaced rectangles.
They can be created by the number of elements in x- or y-direction or vectors of the enclosing nodes:
\begin{lstlisting}
/*! Generate simple 2d mesh with xDim*yDim cells of size 1 */
Mesh createMesh2D( uint xDim, uint yDim );
/*! Generate simple 2d mesh from node vectors x and y */
Mesh createMesh2D( const RVector & x, const RVector & y );
\end{lstlisting}

\subsection*{2d general mesh}
Generally a 2d mesh - a regular one is just a special case - can consist of triangles or quadrangles (deformed rectangles) or a mix of it. 
They are created by mesh generators such as triangle \citep{triangle}.
Input for the mesh is a piece-wise linear complex (PLC) comprising nodes edges and region markers.
Meshing is done externally and loaded using \lstinline|Mesh.load(filename);|.

\subsection*{3d regular mesh}
A regular (FD like) 3d model consists of regularly spaced hexahedra.
They can be created by the number of elements in x/y/z-direction or vectors of the enclosing nodes:
\begin{lstlisting}
/*! Generate regular 3d mesh with xDim*yDim*zDim cells of size 1 */
Mesh createMesh3D( uint xDim, uint yDim, uint zDim );
/*! Generate regular 3d mesh from node vectors x, y and z */
Mesh createMesh3D( const RVector & x, & y, & z );
\end{lstlisting}

\subsection*{3d general mesh}
At the moment, a 3d mesh can consist of tetrahedrons or hexahedrons, but prisms or pyramids could be easily implemented.

%%%%%%%%%%%%%%%%%%%%%%%%%%%%%%%%%%%%%%%%%%%%%%%%%%%%%%%%%%%%%%%%%%%%
\clearpage
\section{Region properties and region map file}\label{app:regions}
Some properties can be set for each region individually using \lstinline|f.region(i).|
\begin{lstlisting}
setMarker( int marker );
setBackground( bool background );
setSingle( bool single );
setStartVector( const RVector & start );
setStartValue( double start );
setModelControl( double mc );
setModelControl( const RVector & mc );
setBoundaryControl( double or RVector bc );
%setZPower( double zpower );
setZWeight( double zweight );
setTransModel( Trans & tM );
setConstraintType( uint type );
setLowerBound( double lb );
setUpperBound( double ub );
\end{lstlisting}

\begin{lstlisting}
uint parameterCount(), boundaryCount(); //! parameters/boundaries
\end{lstlisting}

The region manager controls the individual regions.
It is initialised from the forward operator and interprets the mesh with its markers
\begin{lstlisting}
setMesh( const Mesh & mesh );
Region * createRegion( int marker, const Mesh & mesh ); //! create region
Region * region( int marker ); //! get an individual region by marker
uint parameterCount(), constraintCount(), boundaryCount(); //! counters
RVector createStartVector();  //! create starting model vector
RVector createModelControl(), createBoundaryControl(); //! create vectors
RVector createFlatWeight( double zPower, double minZWeight ); //! zpower
loadMap( const std::string & fname ); //! set region properties from file
\end{lstlisting}

The latter region map file simplifies the setting of region properties and is comfortable for testing different values.
It is a column file with the description of the columns in the first row after a \# sign:
\begin{verbatim}
#No start Ctype MC  zWeight Trans lBound uBound
0   100   1     1   0.2  	  log   50     1000
1   30    0     0.2 1       log   10     200
\end{verbatim}
The example represents a two layer case, e.g. an unsaturated (0) and a saturated (1) zone with different starting resistivities.
Smoothness constraints with enhanced layering is applied in the first and minimum length in the latter.
Both use a log/logLU transformation with specific upper and lower bounds.

Instead of the number, an asterisk (*) can be used to set properties for all regions.
In one region file, several blocks as above can be stated, e.g.
\begin{verbatim}
#No start Trans
*   100   log
#No lBound uBound
1   20     300
2   50     1000
\end{verbatim}
defines an equal model transformation and starting value, but different lower and upper bounds.

There are two special types of regions: background and single regions.
The background region is not part of the inversion, the values are prolongated (filled up) for the forward run.
On the contrary, the cells of a single region are held constant and treated as one parameter in inversion.
They are specified as above using the keywords \verb|background| and \verb|single|, e.g.
\begin{verbatim}
#No single
*	1
#No background
0	1 
\end{verbatim}
defines all regions as single parameter regions except number zero, which is background.

By default, regions are decoupled from each other, i.e. there are no smoothness contraints at the boundary.
However, it might be useful to have those, e.g. by constraining a region of known parameters by borehole data to the neighboring cells or just to stabilize inversion.
In this case, inter-region constraints can be defined in the region file. 
The text
\begin{verbatim}
#Inter-region
* 	*	  0.1
1   2   1
\end{verbatim}
sets weak connection between all regions, except regions 1 and 2 are normally connected.

%%%%%%%%%%%%%%%%%%%%%%%%%%%%%%%%%%%%%%%%%%%%%%%%%%%%%%%%%%%%%%%%%%%%
\clearpage
\section{Transformation functions}\label{app:trans}
For data and model parameter arbitrary transformations can be applied.
A transformation is a C++ class derived from a base class (the identity transformation), which mainly consists of four functions, each returning a vector for a given vector:
\begin{description}
\item[trans] the forward transformation function: $y(x)$
\item[invTrans] the inverse of the function: $x(y)$
\item[deriv] the derivative of the function: $y'(x)$
\item[error] the transformation of associated errors $\delta y(\delta x)$
\end{description}
The latter is defined in the base class.
The inversion as mathematical operation is done in the $y$ domain, whereas the physics is described in the $x$ domain.

Besides the presented transformations, you can define your own transformations by deriving from the base class and overwriting the first three functions.
If the inverse transformation is not known analytically, there is a class {\bf TransNewton}, in which the inverse function is obtained by Newton iteration.

However, there are a lot of already existing transformation classes that can be used or combined:

\subsection*{Basic transforms}
\begin{description}
\item[TransLinear](a,b): $y(x)=a*x+b$
\item[TransPower]($n$): $y(x)=x^n$
\item[TransExp]($x_0$,$y_0$): $y(x)=y_0\cdot e^{-x/x_0}$
\item[TransInv]: $y(x)=1/x$ (specialisation of TransPower)
\item[TransLog]: $y(x)=\log(x)$
\end{description}

\subsection*{Range transforms}
The logarithm restricts $x$ to be positive, i.e. sets a lower bound 0.
Instead of 0, a lower bound $x_l$ can be set.
By combining two logarithmic functions a lower and an upper bound can be combined.
Similar can be obtained by a cotangens function.

\begin{description}
\item[TransLog]($x_l$): $y(x)=\log(x-x_l)$
\item[TransLogLU]($x_l$,$x_u$): $y(x)=\log(x-x_l)-\log(x_u-x)$
\item[TransCotLU]($x_l$,$x_u$): $y(x)=-\cot((x-x_l)/(x_u-x)\cdot\pi)$
\end{description}

\subsection*{Combination}
Different transformations can be combined by either\\
{\bf TransNest}($y^1$,$y^2$): $y(x)=y^1(y^2(x))$\\
{\bf TransAdd}($y^1$,$y^2$): $y(x)=y^1(x)+y^2(x)$\\
Since for the latter the inverse cannot be combined by the two inverses, it is derived from transNewton.
There is a cumulative transformation CumulativeTrans, which applies a vector of transformations for different part of the model.
This meta-transformation is applied in the region technique but can also be defined for one region.
More often it is used to combine different data types, see sections \ref{sec:mt1d} and \ref{sec:jointdcem} for examples.

\subsection*{Special transformations}
Some geophysically relevant transformations have been already defined:
\begin{description}
\item[TransLogMult]($y_0$) is a {\bf TransNest} of {\bf TransLog} and {\bf TransMult}: $y(x)=y_0\log(x_0)$, e.g. for using the logarithm of the apparent resistivity ($\rho^a=G R$)
\item[TransCRIM]($\phi$,$\epsilon_m$,$\epsilon_w$): the complex refractive index model (CRIM) - derived from {\bf TransLinear}
\item[TransArchie]($\rho_w$) - Archie's equation, derived from {\bf TransPower}
\end{description}

Note that there are predefined types based on real (double) vectors beginning with an R, e.g. \lstinline|RTransLogLU| is actually \lstinline|TransLogLU< RVector >|.


\section{Vectors and Matrix types}\label{app:matrix}
\sperre

\lstinline|std::vector|

\lstinline|template< class ValueType > class Vector|

\lstinline|RVector| 

\lstinline|load/save|

complex values using Complex = \lstinline|std::complex<double>| as a vector \lstinline|CVector|

\lstinline|template < class ValueType > class Matrix|

\lstinline|RMatrix|

\lstinline|load/saveMatrixCol|

\lstinline|template< class ValueType, class IndexType > class SparseMapMatrix|

DSparseMapMatrix for double and unsigned int

modelling SparseMatrix